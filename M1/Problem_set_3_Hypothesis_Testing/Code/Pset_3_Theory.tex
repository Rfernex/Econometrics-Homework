\documentclass{article}

\usepackage[utf8]{inputenc}
\usepackage[T1]{fontenc}
\usepackage{amsmath}
\usepackage{amsfonts}
\usepackage{amssymb}
\usepackage[version=4]{mhchem}
\usepackage{stmaryrd}
\usepackage{bbold}
\usepackage{graphicx}
\usepackage{adjustbox}
\usepackage{hyperref}  % For clickable links in TOC
\usepackage{tocloft}   % For TOC formatting
\usepackage{geometry}  % For margins
\usepackage{titlesec}  % For section formatting
\usepackage{fancyhdr}
\usepackage{footmisc}
\usepackage[stable]{footmisc}

\newcommand{\ind}{\perp\!\!\!\!\perp} 
% Document styling
\geometry{margin=1in}

% Customize section formatting
\titleformat{\section}
  {\normalfont\Large\bfseries}
  {}
  {0pt}
  {\thesection\quad\rule{\linewidth}{0.4pt}\vspace{0.5em}\newline}
{\normalfont\Large\bfseries}


% adding a header
\pagestyle{fancy}
\fancyhf{}
\lhead{Problem Set 3 - Theory}
\rhead{Romain Fernex}
\cfoot{ \thepage}
\renewcommand{\headrulewidth}{0.8pt}


\title{PS3 - Theory}
\author{Romain Fernex}
\date{February 2025}

\begin{document}

\maketitle
\tableofcontents


\section{Part II - Theory}

\subsection{Problem 2}

\subsubsection{Question 1 : By what percent is salary predicted to increase, if ros increases by 50 points ?}

The CEO's salary is predicted to increase by 0.00024*50 $\approx$ 1.2\% if ros increases by 50 percentage points


\subsubsection{Question 2 : By what percent is salary predicted to increase if sales increase by 1 percent ?}
The CEO's salary is predicted to increase by 0.28\% if sales increase by 1 percent. 
\begin{itemize}
    \item This interpretation is only valid because the log function is applied to both the dependent and the independent variable. 
\end{itemize}

\subsubsection{Question 3 : Test the null hypothesis that ros has no effect on salary, against the alternative that ros has a positive effect. The critical value at the 10 percent significance level is 1.282.}
\textbf{Set up of the test : }one-sided test 
\begin{equation}
\begin{aligned}
    H_0 : \beta_3 = 0\\
    H_1 : \beta_3 >0
\end{aligned}
\end{equation}
Let's note W the test statistic. We reject $H_0$ only if W is superior to the critical value at the 10\% significance level.\\ 
\textbf{Value of the test statistic : } we note $\hat{T}$ the estimator for the parameter we wish to estimate (here $\beta_3)$
\begin{equation}
    W = \frac{\hat{T}-0}{SE(\hat{T})} = \frac{2.4*10^{-4}}{5.4*10^{-4}} = 0.44
\end{equation}
We know the critical value $c_{10\%}$ is equal to 1.282. Thus we find that W < 1.282.
We cannot reject the null hypothesis at the 10 \% significance level, which means it is unlikely that the ros has an effect on salary. 

\subsubsection{Question 4 : Would you include ros in a final model explaining CEO compensation in terms of firm performance? Explain.}
I would not include the ROS in the final model as it lacks statistical significance (see question 3) and since dropping it might help simplify the model further.
That being said, comparing the explanatory power of the model before and after removing this variable (by looking at the adjusted R-squared) would be useful to see if this variable indeed contributed little to it. 

\subsection{Problem 3}
We know that, for $f(\theta) = \gamma = (b,s)$, we have : 
\begin{equation}
    \sqrt{n}(\hat{\gamma}-\gamma) \xrightarrow{d} N(0,\Sigma)
\end{equation}
We wish to find an estimator for $\theta$. Let us take the following transformation : $g(\gamma) = (\frac{b}{s},\frac{1}{s})= \theta$
Using the Delta method, we obtain the following:
\begin{equation}
    \sqrt{n}(g(\hat{\gamma})-g(\gamma)) \xrightarrow{d} J(\gamma)N(0,\Sigma) \text{ with }J(\gamma)\text{ the Jacobian matrix}
\end{equation}
Let us compute $J(\gamma)$: 
\begin{equation}
    J(\gamma) = \begin{pmatrix}
        \frac{\delta g_1(\gamma)}{\delta b} & \frac{\delta g_1(\gamma)}{\delta s} \\
        \frac{\delta g_2(\gamma)}{\delta b} & \frac{\delta g_2(\gamma)}{\delta s}
    \end{pmatrix} = \begin{pmatrix}
        \frac{1}{s} & -\frac{b}{s^2}\\
        0 & -\frac{1}{s^2} \end{pmatrix} = \begin{pmatrix}
            \sigma & -\beta \sigma\\
            0 & -\sigma^2
    \end{pmatrix}
\end{equation}
Before computing the variance, let us find what $\hat{\theta}$ tends to in probability. Knowing $g(\hat{\gamma}) = \hat{\theta}$ and $g(\gamma) = \theta$, based on equation (4) we have : 
\begin{equation}
\begin{aligned}
    \sqrt{n}(\hat{\theta} - \theta) \xrightarrow{d} N(0,J(\gamma)\Sigma J(\gamma)^T) \\
    \iff \hat{\theta} \xrightarrow{d} N(\theta, \frac{1}{n}J(\gamma)\Sigma J(\gamma)^T)
\end{aligned}
\end{equation}
We can now express $V(\hat{\theta})$ as a function of $\sigma$ and $\beta$ :
\begin{equation}
\begin{aligned}
    V(\hat{\theta}) &= \frac{1}{n}\begin{pmatrix}
        \sigma & -\beta\sigma\\
        0 & -\sigma^2
    \end{pmatrix} \begin{pmatrix}
        \sigma_1^2 & w\\
        w & \sigma_2^2
    \end{pmatrix} \begin{pmatrix}
        \sigma & 0\\
        -\beta\sigma & -\sigma^2
    \end{pmatrix}\\
    &= \frac{1}{n}\begin{pmatrix}
        \sigma^2\sigma_1^2+\beta^2\sigma^2\sigma_2^2 - 2 \beta\sigma^3w & -\beta\sigma^2\sigma_2^2+\sigma^3w\\
        -\beta\sigma^2\sigma_2^2+\sigma^3w & \sigma^4\sigma_2^2
    \end{pmatrix}
\end{aligned}    
\end{equation}





\end{document}
